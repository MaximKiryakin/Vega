\documentclass[10pt,a4paper]{article}
\usepackage[12pt]{extsizes}
\usepackage[utf8]{inputenc}
\usepackage[russianb]{babel}
\usepackage[left=1.0cm,right=1.0cm,top=0.5cm,bottom=0.5cm]{geometry}
\usepackage{setspace}
\usepackage{indentfirst}
%\usepackage{amssymb}
%\usepackage{amsmath}
%\usepackage{array}
\usepackage{tabularx}
\usepackage{multicol}
\usepackage[pdftex]{graphicx}
\usepackage{comment}


\usepackage{hyperref}

\usepackage{titling}  
\usepackage{geometry} 
\usepackage{fancyhdr}


\setlength{\headheight}{15mm}


\pagestyle{fancy}

\lhead{\textbf{\normalsize Проект 1}}
\rhead{\textbf{\normalsize Выполнили: }\normalsize Кирякин Максим, Куренкова Дарья, Коваль Наталья}



\begin{document}
	

\section{Постановка задачи}

В работе было необходимо построить прогноз значений временного ряда с помощью моделей эконометрики. 

Модель ARIMA (Autoregressive Integrated Moving Average) используется для прогнозирования временных рядов.
Формула модели ARIMA(p, d, q):
$$
\Delta^d y_t = \alpha_1 \Delta^d y_{t-1} + ... + \alpha_p\Delta^dy_{t-p} + \varepsilon_t + \beta_1\varepsilon_{t-1} + ... + \beta_q\varepsilon_{t-q},
$$

$y_t$ - значение временного ряда в момент времени $t$, $\Delta = (1 - L)^d$, $\varepsilon_t$ - белый шум. L - лаговый оператор.

Модель SARIMA – расширение для рядов с сезонной составляющей. SARIMAX – расширение, включающее внешнюю регрессионную составляющую


\section{Ход работы}
В работе решалась задача прогнозирования числа продаж в сети магазинов Эквадора. Было рассмотрено 3 способа прогнозирования:
\begin{itemize}
	\item ARIMA модель с ручной настройкой параметров p, d и q
	\item SARIMAX модель
	\item Модель машинного обучения (градиентный бустинг)
\end{itemize}


Качество оценивалось с помощью следующих метрик: MSE, RMSE, MAE, абсолютная процентная ошибка. 
Результаты сравнивались с наивным прогнозом -- средним числом продаж за последнюю четверть тренировочного периода. Выбор параметров для ARIMA модели осуществлялся на основе значений функций автокорреляции и частичной автокорреляции. Стационарность ряда проверялась с помощью теста  Дики — Фуллера.

Кроме этого, в работе была обучена ml-модель градиентного бустинга. Для нее был использован перебор гиперпараметров пакета hyperopt, в основе которого лежит байесовская оптимизация. Результаты бустинга сравнивались с результатами экономических моделей и наивного прогноза. 

Все графики и данные доступны на странице  
\href{https://github.com/MaximKiryakin/Vega/tree/main/%D0%A4%D0%B8%D0%BD%D0%B0%D0%BD%D1%81%D0%BE%D0%B2%D0%B0%D1%8F%20%D1%8D%D0%BA%D0%BE%D0%BD%D0%BE%D0%BC%D0%B5%D1%82%D1%80%D0%B8%D0%BA%D0%B0/%D0%9F%D1%80%D0%BE%D0%B5%D0%BA%D1%82%201}{проекта}.

\section{Заключение}

 Были получены следующие значения метрик:


\begin{table}[h]
	\centering
	\caption{Преимущества и недостатки недифференцированного маркетинга}
	\begin{tabularx}{\textwidth}{|X|X|X|X|X|}
		\hline
		Метрика & ARIMA & SARIMAX & XGBRegressor & Наивный прогноз \\
		\hline
		MSE & 370425.063612 & 11849.563053 & 2761.131836 & 20720.685499 \\
		RMSE & 608.625553 & 108.855698 & 52.546473 & 143.946815 \\
		MAE & 600.125711 & 70.474641 & 42.322258 & 114.795845 \\
		Процентная ошибка & 1.133873 & 0.117918 & 0.080065 & 0.196057 \\
		\hline
	\end{tabularx}
\end{table}



По метрикам RMSE, MAE и процентной ошибке лучше всего себя показала ML модель. Второй по качеству оказалась SARIMAX модель. Полученные результаты можно объяснить тем, что ML модель имеет другую архитектуру (более сложный и тонкий процесс обучения), является более совершенной и новой, по сравнению с эконометрической моделью.
	
	
\end{document}
